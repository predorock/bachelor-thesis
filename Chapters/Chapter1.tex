\chapter{Stato dell'arte e obiettivi della tesi}
In questo capitolo si va a presentare il lavoro di ricerca sulle tecnologie web per lo sviluppo di applicazioni cross-platform svoltosi presso la ditta BigThink SRL e sgli strumenti che rappresentano al meglio lo stato dell'arte attuale. La fine del capitolo descriverà gli obiettivi di questa ricerca.

\section{Big Think SRL}
L'azienda BigThink SRL di Daniele Ghidoli è una startup che si occupa dello sviluppo di applicazioni web di vario genere. Durante la mia permanenza nell'azienda ho lavorato a applicazioni pubblicitarie, a volte nel mercato dei social media e ho sviluppato uno strumento interno all'azienda per la gestione dei progetti da parte dei dipendenti. 
\subsection{Il mio ruolo all'interno dell'azienda}
Il mio ruolo si può descrivere come Frontend Developer ovvero sviluppavo a lato client l'interfaccia dell'applicazione che interagiva con l'utente. 
\section{Lo stato dell'arte}
Grazie alla formazione ottenuta presso l'azienda sono partito con un bagaglio formativo molto adeguato per svolgere la ricerca. Durante il lavoro potevo usufruire della collaborazione del mio correlatore e del capo azienda.
\subsection{AngularJS}
AngularJS è un framework javascript che consente di estendere il vocabolario dell'HTML. Viene definito il "supereroe di Google" e uno dei vantaggi più grandi che caratterizzano questo framework è la possibilità di integrare e utilizzare molte funzioni utilizzando quasi esclusivamente l’HTML grazie all'approccio dichiaritivo. Questo rende molto facile l'interazione con il framework anche ai meno esperti; un altro punto a favore di Angular è il fatto che grazie a questa semplicità di utilizzo si ottengono già grandi risultati e si ha la possibilità, in pochissimo tempo, di rendere il contenuto dinamico via variabili definite in Javascript.
\subsection{Grunt e Bower}
Ogni sviluppatore ha degli strumenti che lo accompagnano nello sviluppo di codice; in particolare io mi sono munito di un package manager e di un task-runner
\paragraph{Grunt}
Grunt è appunto il nostro task runner. Nell'organizzazione del codice è buona pratica separare parti del progetto in file diversi, suddivise per moduli o in base all'utilità semantica che hanno e poi concatenare tutto quello che abbiamo fatto. Inoltre alla durante lo sviluppo e/o dopo lo sviluppo vogliamo eseguire dei test su ciò che abbiamo scritto. GRUNT ci consente tutto questo; grazie alla sua struttura modulare possiamo scegliere le procedure che riteniamo più adatte per rendere il progetto dinamico e ricercare in maniera mirata i bug presenti nel codice.
\paragraph{Bower}
Nella nostra applicazione spesso vogliamo includere librerie esterne che fanno determinate cose in modo da non scrivercele tutte le volte a mano. Bower ci permette con due comandi di includere dinamicamente qualsiasi libreria presente su Github e di posizionarla dove riteniamo più opportuno all'interno della struttura del nostro progetto.  
\section{Obiettivi del lavoro}
Ho voluto pormi gli obiettivi di questa ricerca dando uno sguardo alla realtà del mondo del lavoro che si sta rapidamente evolvendo in questa fetta del mercato delle applicazioni web.
\subsection{La richiesta del mercato}
\subsubsection{Ciclo di vita di una app}
Ci siamo posti la domanda di quanto debba durare la vita di una applicazione in generale. Sicuramente se si vuole mantenere la propria applicazione all'interno dello store di riferimento bisogna tenere conto dell'evolversi delle tecnologie e quindi preoccuparsi degli aggiornamenti alle versioni dei linguaggi funzionanti. 
\subsubsection{Più mercati su cui competere}
Inoltre dobbiamo tenere conto che ci sono diversi mercati di applicazioni su cui competere, i quali utilizzano linguaggi differenti che a sua volta devono essere mantenuti e aggiornati.
Un altro problema legato alla concorrenza su mercati differenti, è il costo di produzione delle diverse applicazioni in linguaggi diversi; implica avere a disposizione più sviluppatori con diverse competenze e quindi ad aumentare le spese di produzione.
\subsubsection{Traffico internet da Audiweb}
-- Dati rilevanti sul traffico Mobile
\subsection{Modelli di sviluppo}
Come abbiamo visto dai dati precedenti il traffico internet proveniente dai dispositivi mobili e nettamente in crescita. Lo scopo di questa tesi si occuperà di ricercare soluzioni per lo sviluppo di applicazioni multi-piattaforma per dispositivi mobili.

Per fare ciò abbiamo bisogno di vedere quali modelli di sviluppo si adattano alle nostre esigenze e competenze.

\subsubsection{Sviluppo Nativo}
Nello sviluppo nativo di una applicazione dobbiamo tenere conto come detto in precedenza che esistono più mercati(Store di applicazioni) dove una eventuale azienda potrebbe decidere di concentrare la sua applicazione. Ecco le principali piattaforme di sviluppo al momento:

\begin{itemize}
\item Android, Java
\item iOS, Objective C / Swift
\item Windows Phone, .NET
\item Blackberry, C++/Qt
\item FirefoxOS, HTML5, CSS, Javascript
\item Tizen, HTML5, CSS, Javascript
\end{itemize}

\subsubsection{Le web app}
Una realtà che si sta via via prendendo sempre più piede nello sviluppo delle applicazioni e quella delle Web App. Si possono definire Rich Internet Application (RIA) ed hanno tutte le caratteristiche di un programma che normalmente è in esecuzione sul nostro desktop con la differenza che anziché essere installate nel sistema operativo vengono eseguite nel browser web. 

I vantaggi di questo approccio sono: abbiamo la nostra Applicazione in esecuzione in una Sandbox separata dal resto del sistema operativo ovvero il browser web(che può essere generico, non ne serve uno specifico), non dobbiamo preoccuparci degli aggiornamenti (basta aggiornare la pagina nel browser), non dobbiamo imparare a sviluppare in linguaggi e paradigmi diversi, ci bastano HTML5, CSS e Javascript.

Dall'altro lato della medaglia siamo limitati sotto certi aspetti: l'accesso alle funzionalità del sistema operativo è molto limitato, inoltre la memorizzazione di dati risulta molto difficile da implementare(se cancelliamo la cache del browser perdiamo tutti i dati della nostra applicazione). Ma il più grande deficit è che la Web App per funzionare deve essere sempre connessa ad Internet. 
\subsection{La ricerca della soluzione}  
Abbiamo visto alcuni dei vantaggi e degli svantaggi dei vari tipi di approccio allo sviluppo, è abbiamo questa situazione:
\begin{center}
	\begin{tabular}{| l || c | c | }
		\hline
							&	Native App	&	Web App       \\
		\hline
		\hline
			Cross Platform?	&	No			&	Si            \\
		\hline
			Accesso alle api 
			del sistema 
			operativo?		&	Si			&	Più no che si \\
		\hline
			Store di dati	&	Si			&	Più no che si \\
		\hline
			Connessione 
			obbligatoria ?	&	No			&	Si 			  \\
		\hline
			Velocità di 
			esecuzione	    &	Ottima	 	& 	Buona \\
		\hline
			Tecnologia di 
			sviluppo		&	Objective C, 
			  					Java, C++,..	&	HTML, CSS, Javascript \\
		\hline
			Update 
			installabili?	&	Si			&	Non servono!! \\	
		\hline		
	\end{tabular}
\end{center}
Continuando le ricerche ho scoperto che la scelta migliore non è nei due approcci, ma sta nel mezzo.
Le \emph{Native Web Application} sono la soluzione discussa in questa tesi, e combinano i fattori positivi dei 2 approcci appena visti, con qualche cambiamento.
 