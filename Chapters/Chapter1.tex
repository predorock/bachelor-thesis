\lhead{\emph{Capitolo 1}}
\chapter{Stato dell'arte} 
In questo capitolo si presentano le tecnologie attualmente presenti sul mercato per lo sviluppo di applicazioni mobili, con particolare attenzione per le strategie di sviluppo multi piattaforma.
\section{Sviluppo Nativo}
Nello sviluppo nativo ogni applicazione viene sviluppata singolarmente per ogni piattaforma; rispettivamente per ogni sistema operativo bisogna conoscere i linguaggi che vengono utilizzati.

\csvautotabular{Primitives/platforms.csv}

Il punto di forza che ha questo tipo di approccio sta sicuramente nelle prestazioni dell'applicazione e nel completo accesso a tutte le possibilità del nostro dispositivo. Tuttavia con questo tipo sviluppo richiede molto tempo nel caso della distribuzione su più piattaforme.
\section{Multi-Piattaforma}

\subsection{Web Applications}
Quando si intraprende la strada dello sviluppo di una applicazione multi-piattaforma l'opzione più semplice è sicuramente quella di creare una Web App responsive\footnote{indica la capacità di una Web App o di un sito internet di adattare le proprie dimensioni e proporzioni in base alla risoluzione del dispositivo con il quale vengono visualizzati} che sia accessibile in modo immediato dal browser web del nostro dispositivo. In questo caso i tempi di sviluppo sono veramente esigui, e non abbiamo bisogno di aggiornamenti in quanto si tratta alla fine di un piccolo sito internet(quindi ci basta aggiornare la pagina).

A scapito della velocità di produzione e della facilità di aggiornamento non possiamo permetterci alcune caratteristiche fondamentali per le applicazioni mobile, come ad esempio accedere alle funzionalità del nostro dispositivo come fotocamera o rubrica, oppure memorizzare dati sul nostro dispositivo(o meglio possiamo farlo ma sfruttando solamente la cache del browser web) ma la criticità maggiore sta nel fatto che la connessione internet deve essere sempre presente. 

\subsection{I Framework}
L'approccio ibrido che si propone combina la velocità di sviluppo delle Web App con un accesso \emph{quasi nativo} al nostro dispositivo. Un problema che ci so pone in questo tipo di approccio sta nella scelta del \emph{framework} che si adatta  meglio allo scopo della nostra applicazione.

\textit{In informatica, e specificatamente nello sviluppo software, un \textbf{framework} è un'architettura (o più impropriamente struttura) logica di supporto (spesso un'implementazione logica di un particolare design pattern) su cui un software può essere progettato e realizzato, spesso facilitandone lo sviluppo da parte del programmatore. Alla base di un framework c'è sempre una serie di librerie di codice utilizzabili in fase di linking con uno o più linguaggi di programmazione, spesso corredate da una serie di strumenti di supporto allo sviluppo del software, come ad esempio un IDE, un debugger o altri strumenti ideati per aumentare la velocità di sviluppo del prodotto finito. L'utilizzo di un framework impone dunque al programmatore una precisa metodologia di sviluppo del software.}\\
\hspace*{\fill}\citeauthor{wiki:framework}

Esistono diversi framework sul mercato con caratteristiche diverse tra di loro, ecco quelli più popolari:

\csvautotabular{Primitives/frameworks.csv}

Ogni framework adotta strategie diverse per passare dal codice sorgente a quello di destinazione, successivamente verranno discusse alcune di esse.

\subsubsection{Wrapper}
I framework denominati appunto \texttt{Wrapper} sfruttano il design pattern strutturale \texttt{Adapter} e operano quindi una astrazione sul linguaggio di destinazione. In questo caso il framework ha il compito di intermediario tra il linguaggio usato nella applicazione e quello nativo del dispositivo.
Cordova ad esempio simula un browser web sul linguaggio di astrazione(che in questo caso saranno linguaggi in ambito web) e fornisce delle API per comunicare con la parte nativa del linguaggio del dispositivo.
La game-engine \texttt{Unity} utilizza lo stesso meccanismo si astrazione, ed anziché simulare un browser utilizza un linguaggio proprietario per poter eseguire la applicazione su più piattaforme.
Un altro esempio è invece Apache Flex che utilizza come linguaggio di astrazione Adobe Flash e in alcuni casi Javascript.
\subsubsection{Trasformazione del Codice}
Alcuni framework potremmo definirli invece \emph{"trasformatori"} ovvero partendo da un loro linguaggio / metalinguaggio(spesso somigliante ad altri linguaggi conosciuti come ad esempio Java) producono un il codice specifico per ogni sistema operativo scelto.
In particolare il linguaggio utilizzato dal framework è una associazione tra il suo linguaggio è quello di destinazione. Una volta creata e riconosciuta l'associazione, il codice viene trasformato.
\subsubsection{User Interface}
Altri tipi di framework invece sono atti alla definizione dell'interfaccia utente della nostra applicazione, ovvero si traducono in un insieme di librerie che utilizzano metalinguaggi e non per \emph{"vestire"} la nostra applicazione a seconda dello scopo.
Un esempio di un framework di questo tipo è \texttt{Ionic}, il quale fornisce un set di stili e funzioni per la nostra applicazione utilizzando tecnologie web, come ad esempio menù laterali, bottoni, form, icone. La maggior parte di questi tipi di framework fornisce anche linee guida per creare elementi completamente nuovi, nel caso avessimo bisogno di parti più complesse.
Invece \texttt{Apache Flex}, oltre a fungere da wrapper, fornisce un set di elementi già pronti e parametrizzati (il linguaggio è simile a XML\footnote{(sigla di eXtensible Markup Language) è un linguaggio di markup, ovvero un linguaggio marcatore basato su un meccanismo sintattico che consente di definire e controllare il significato degli elementi contenuti in un documento o in un testo.\citeauthor{wiki:xml}}) per poter personalizzare la nostra applicazione. Non è molto versatile ma è particolarmente adatto alle applicazioni nel campo produttivo/finanziario.
\section{Vantaggi e Svantaggi}
Il vantaggio principale nell’utilizzo di un framework consiste nella possibilità di effettuare un deploy rapido e multipiattaforma di un’applicazione: il codice, infatti, si scrive velocemente con un meta linguaggio ed attraverso librerie esistenti precompilate, ed esso viene automaticamente adattato per i diversi sistemi operativi in fase di compilazione.

Un limite dei framework consiste nell’eccesso di rigidità in fase progettuale: il loro utilizzo, infatti, limita in parte la possibilità di sfruttare le caratteristiche native del device, delle linee guida dell’interfaccia e del sistema operativo che utilizzerà l’applicazione, e spesso, quindi, si corre il rischio di non riuscire a sviluppare interfacce utente pienamente ergonomiche e intuitive.
Inoltre, alcuni framework sono più adatti ad impieghi specifici rispetto ad altri. Kony, ad esempio, è un tipo di framework particolarmente apprezzato nello sviluppo di applicazioni per il settore bancario e financial, poichè integra al suo interno numerose funzioni utili,tra cui per la gestione dei sistemi di pagamento attraverso gli smartphone.

Un tipo di framework scelto nel contesto giusto, che tenga concretamente conto degli obiettivi di marketing dell’azienda e le strategie di investimento in sviluppo di prodotti, consente di ottimizzare i tempi di sviluppo, presentare velocemente sul mercato la propria applicazione ed ottimizzare il processo di delivery abbattendo costi e sfruttando le potenzialità specifiche del framework selezionato.\\
\hspace*{\fill}\citep{web:framework}

