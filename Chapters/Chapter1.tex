\chapter{Stato dell'arte e obiettivi della tesi}
In questo capitolo si va a presentare il lavoro di ricerca sulle tecnologie web per lo sviluppo di applicazioni cross-platform svoltosi presso la ditta BigThink SRL e sgli strumenti che rappresentano al meglio lo stato dell'arte attuale. La fine del capitolo descriverà gli obiettivi di questa ricerca.

\section{Big Think SRL}
L'azienda BigThink SRL di Daniele Ghidoli è una startup che si occupa dello sviluppo di applicazioni web di vario genere. Durante la mia permanenza nell'azienda ho lavorato a applicazioni pubblicitarie, a volte nel mercato dei social media e ho sviluppato uno strumento interno all'azienda per la gestione dei progetti da parte dei dipendenti. 
\subsection{Il mio ruolo all'interno dell'azienda}
Il mio ruolo si può descrivere come Frontend Developer ovvero sviluppavo a lato client l'interfaccia dell'applicazione che interagiva con l'utente. 
\section{Lo stato dell'arte}
Grazie alla formazione ottenuta presso l'azienda sono partito con un bagaglio formativo molto adeguato per svolgere la ricerca. Durante il lavoro potevo usufruire della collaborazione del mio correlatore e del capo azienda.
\subsection{AngularJS}
AngularJS è un framework javascript che consente di estendere il vocabolario dell'HTML. Viene definito il "supereroe di Google" e uno dei vantaggi più grandi che caratterizzano questo framework è la possibilità di integrare e utilizzare molte funzioni utilizzando quasi esclusivamente l’HTML grazie all'approccio dichiaritivo. Questo rende molto facile l'interazione con il framework anche ai meno esperti; un altro punto a favore di Angular è il fatto che grazie a questa semplicità di utilizzo si ottengono già grandi risultati e si ha la possibilità, in pochissimo tempo, di rendere il contenuto dinamico via variabili definite in Javascript.
\subsection{Grunt e Bower}
Ogni sviluppatore ha degli strumenti che lo accompagnano nello sviluppo di codice; in particolare io mi sono munito di un package manager e di un task-runner
\paragraph{Grunt}
Grunt è appunto il nostro task runner. Nell'organizzazione del codice è buona pratica separare parti del progetto in file diversi, suddivise per moduli o in base all'utilità semantica che hanno e poi concatenare tutto quello che abbiamo fatto. Inoltre alla durante lo sviluppo e/o dopo lo sviluppo vogliamo eseguire dei test su ciò che abbiamo scritto. GRUNT ci consente tutto questo; grazie alla sua struttura modulare possiamo scegliere le procedure che riteniamo più adatte per rendere il progetto dinamico e ricercare in maniera mirata i bug presenti nel codice.
\paragraph{Bower}
Nella nostra applicazione spesso vogliamo includere librerie esterne che fanno determinate cose in modo da non scrivercele tutte le volte a mano. Bower ci permette con due comandi di includere dinamicamente qualsiasi libreria presente su Github e di posizionarla dove riteniamo più opportuno all'interno della struttura del nostro progetto.  
\section{Obiettivi del lavoro}
Ho voluto pormi gli obiettivi di questa ricerca dando uno sguardo alla realtà del mondo del lavoro che si sta rapidamente evolvendo in questa fetta del mercato delle applicazioni web.
\subsection{La richiesta del mercato}
\subsubsection{Ciclo di vita di una app}
Ci siamo posti la domanda di quanto debba durare la vita di una applicazione in generale. Sicuramente se si vuole mantenere la propria applicazione all'interno dello store di riferimento bisogna tenere conto dell'evolversi delle tecnologie e quindi preoccuparsi degli aggiornamenti alle versioni dei linguaggi funzionanti. 
\subsubsection{Più mercati su cui competere}
Inoltre dobbiamo tenere conto che ci sono diversi mercati di applicazioni su cui competere, i quali utilizzano linguaggi differenti che a sua volta devono essere mantenuti e aggiornati.
Un altro problema legato alla concorrenza su mercati differenti, è il costo di produzione delle diverse applicazioni in linguaggi diversi; implica avere a disposizione più sviluppatori con diverse competenze e quindi ad aumentare le spese di produzione.
\subsubsection{Traffico internet da Audiweb}
-- Dati rilevanti sul traffico Mobile
\subsection{Problemi legati alle richieste}
Come ben sappiamo a seconda dei sistema operativo del dispositivo mobile, il linguaggio di programmazione cambia. Inoltre anche i modelli di sviluppo sono differenti, quindi se vogliamo portare una applicazione su x mercati dobbiamo usare x linguaggi di programmazione e x modelli di sviluppo differenti.
\begin{itemize}
\item Android, Java
\item iOS, Objective C / Swift
\item Windows Phone, .NET
\item Blackberry, C++/Qt
\item FirefoxOS, HTML5, CSS, Javascript
\item Tizen, HTML5, CSS, Javascript
\end{itemize}
 
-- Più mercati su cui competere = Più linguaggi da imparare --
-- Riuso del codice non possibile -- 
-- Modelli di sviluppo differenti --
 * chi più ne ha più ne metta *
\subsection{La ricerca della soluzione}  
-- Linguaggi portabili --
-- Cross Platform --
 