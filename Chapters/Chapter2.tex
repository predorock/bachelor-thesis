\chapter{Architettura di una Applicazione Ibrida}
In questo capitolo vedremo alcuni concetti utili per lo sviluppo di applicazioni ibride e come sono strutturate.
\section{Frontend e Backend}
Queste due parole sono spesso usate in informatica in molti ambiti, nel contesto specifico dell'applicazione \texttt{frontend}(in italiano parte davanti) denota quella parte dell'applicazione responsabile di gestire l'interfaccia utente e i dati provenienti da essa, mentre \texttt{backend}(in italiano parte dietro) indica la sezione dell'applicazione dedita alla gestione dei dati provenienti dalla parte frontend. L'interazione che hanno le due parti è un chiaro esempio di interfaccia.\\
Spesso nella parte backend si integra una connessione ad un database per una eventuale memorizzazione di dati, questo approccio ha dei vantaggi fin da subito a lato della sicurezza in quanto l'accesso ai dati non e permesso alla parte frontend la quale può usufruire solo delle funzionalità che la parte backend offre.
\section{Pattern Client-Server}
Il pattern \texttt{client-server} nel contesto delle reti si riduce alla descrizione di un processo comunicativo in rete in cui un host\footnote{termine che indica un nodo generico nella rete} richiede una risorsa, appunto client, mentre l'entità che fornisce la risorsa a più client è detta server. \\
Questa struttura comunicativa per lo scambio di risorse negli anni è stata adattata a molti contesti informatici, ormai non descrive più in particolare la comunicazione nelle reti ma più in generale lo scambio di risorse tra più infrastrutture.\\
Nel capitolo precedente abbiamo descritto frontend e backend, quello è un esempio emblematico di una comunicazione client-server non specificatamente nell'ambito delle reti.
\section{Struttura e Framework}
Abbiamo capito che per la progettazione di una applicazione ibrida
Nella progettazione di una applicazione ibrida bisogna fare delle scelte che sostanzialmente consistono nella scelta dei framework, uno che definisca l'interfaccia utente e la logica dell'applicazione e un altro che si occupi di fornire l'interazione con il sistema operativo.\\
Dall'interfaccia comunicative che si viene a creare possiamo definirli rispettivamente frontend framework e backend framework.

Quello che ora possiamo definire \texttt{frontend framework} si occupa sostanzialmente di fornire una libreria di funzioni per lo sviluppo dell'interfaccia utente e della logica dell'applicazione. Nel momento in cui si necessità di una funzionalità nativa del device si deve utilizzare l'interfaccia fornita dal \texttt{backend framework}.

-- Livelli di una applicazione -- \\
-- Ruolo dei UI Framework - Frontend \\
	-- framework intesi solo come librerie --\\
-- Ruolo dei wrapper - Api -Backend \\
-- Riepilogo delle scelte da fare per i componenti di una applicazione ibrida --

\section{Ciclo di vita di una app}
------ Non sono proprio sicuro di metterlo qui,intanto butto giù il contenuto -----\\
-- Statistiche sulle app\\
	-- Stessa App replicata nei store --\\
	-- Durata delle app --\\
-- Aggiornamenti --\\
	-- più piattaforme = più aggiornamenti --\\
-- Mantenimento --\\
	-- Aggiornamento delle versioni --\\
	-- Debug --\\
-- Tutto ovviamente in confronto con lo sviluppo ibrido --

	
