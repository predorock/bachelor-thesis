\chapter{Ricerca degli strumenti adatti per lo sviluppo}
Prima dello svolgimento della ricerca non ero a conoscenza di quali tecnologie avrei incontrato o utilizzato. Soltanto dopo ho capito come utilizzarle e classificarle per poter scegliere quelle più adatte.
\section{Come rendere una applicazione Cross Platform}
\subsection{Cordova}
\subsection{Titanium}
\subsection{Monaca}
\section{Design e Struttura dell'Applicazione}
\subsection{Struttura dell'App}
-- Ionic --
-- Foundation --
-- Famo.us --
-- Lungo --
-- Monaca UI --
\section{La scelta degli strumenti}
\subsection{AngularJS}
AngularJS è un framework javascript che consente di estendere il vocabolario dell'HTML. Viene definito il "supereroe di Google" e uno dei vantaggi più grandi che caratterizzano questo framework è la possibilità di integrare e utilizzare molte funzioni utilizzando quasi esclusivamente l’HTML grazie all'approccio dichiaritivo. Questo rende molto facile l'interazione con il framework anche ai meno esperti; un altro punto a favore di Angular è il fatto che grazie a questa semplicità di utilizzo si ottengono già grandi risultati e si ha la possibilità, in pochissimo tempo, di rendere il contenuto dinamico via variabili definite in Javascript. 
\subsection{Grunt e Bower}
Ogni sviluppatore ha degli strumenti che lo accompagnano nello sviluppo di codice; in particolare io mi sono munito di un package manager e di un task-runner
\paragraph*{Grunt}
Grunt è appunto il nostro task runner. Nell'organizzazione del codice è buona pratica separare parti del progetto in file diversi, suddivise per moduli o in base all'utilità semantica che hanno e poi concatenare tutto quello che abbiamo fatto. Inoltre alla durante lo sviluppo e/o dopo lo sviluppo vogliamo eseguire dei test su ciò che abbiamo scritto. GRUNT ci consente tutto questo; grazie alla sua struttura modulare possiamo scegliere le procedure che riteniamo più adatte per rendere il progetto dinamico e ricercare in maniera mirata i bug presenti nel codice.
\paragraph*{Bower}
Nella nostra applicazione spesso vogliamo includere librerie esterne che fanno determinate cose in modo da non scrivercele tutte le volte a mano. Bower ci permette con due comandi di includere dinamicamente qualsiasi libreria presente su Github e di posizionarla dove riteniamo più opportuno all'interno della struttura del nostro progetto.
\subsection{Ambiente di sviluppo}

