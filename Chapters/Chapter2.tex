% Chapter Template

\chapter{HTML, CSS, Javascript - L'evoluzione dei linguaggi web} % Main chapter title

\label{ChapterX} % Change X to a consecutive number; for referencing this chapter elsewhere, use \ref{ChapterX}

\lhead{Capitolo 2. \emph{HTML, CSS, Javascript - L'evoluzione dei linguaggi web}} % Change X to a consecutive number; this is for the header on each page - perhaps a shortened title

Linguaggi come l'HTML(Hypertext Markup Language), CSS(Cascading Style Sheet) e Javascript sono i principali strumenti con quali costruiamo pagine web, ebbene la continua evoluzione di questi linguaggi ha portato il classico sito web, a somigliare molto ad una applicazione vera e propria. 
Questo però non sarebbe stato possible se  questi tre linguaggi on si fosssero evoluti parallelamente.
\section{L'HTML - Da descrittore di documenti a descrittore di interfacce}
Nato in origine per la descrizione dei primi documenti reperibili sul web, lo standard HTML è diventato ben presto HTML5, con la possibilità di offrire all'utente un' interfaccia dinamica con cui interagire.
Guardando ai giorni nostri le caratteristiche più significative che vogliamo far notare sono sicuramente il supporto ai contenuti multimediali e la possibilità di creare presentazioni grafiche molto accativanti
arricchendo il codice con i CSS.

\section{CSS - Dallo stile dei documenti alle trasformazioni vettoriali}
Il linguaggio CSS è servito fin dalla sua origine alla descrizione dello stile di un documento HTML per separarlo dal suo contenuto in modo che il codice risultasse più leggibile e riutilizzabile.
L’introduzione del CSS si è resa necessaria per separare i contenuti dalla formattazione e permettere una programmazione più chiara e facile da utilizzare, sia per gli autori delle pagine HTML che per gli utenti, garantendo contemporaneamente anche il riuso di codice ed una sua più facile manutenibilità.
L'ultimo standard CSS3 ha introdotto inoltre nuve librerie grafiche che si appoggiano sul motore di rendering del browser(Blink, Trident, Presto, Gecko) che permettono animazioni e modifiche molto avanzate dei componenti web.
L'unica nota negativa è che ogni motore di rendering utilizza prefissi diversi nelle varie funzioni (-moz ,-webkit). Quindi se si vuole la stessa elaborazione su browser differenti bisogna scrivere la stessa operazione con prefissi differenti.

\section{Javascript - Dalle animazioni ai potenti Framework}
Morbi rutrum odio eget arcu adipiscing sodales. Aenean et purus a est pulvinar pellentesque. Cras in elit neque, quis varius elit. Phasellus fringilla, nibh eu tempus venenatis, dolor elit posuere quam, quis adipiscing urna leo nec orci. Sed nec nulla auctor odio aliquet consequat. Ut nec nulla in ante ullamcorper aliquam at sed dolor. Phasellus fermentum magna in augue gravida cursus. Cras sed pretium lorem. Pellentesque eget ornare odio. Proin accumsan, massa viverra cursus pharetra, ipsum nisi lobortis velit, a malesuada dolor lorem eu neque.

%----------------------------------------------------------------------------------------
%	SECTION 2
%----------------------------------------------------------------------------------------

\section{Main Section 2}

Sed ullamcorper quam eu nisl interdum at interdum enim egestas. Aliquam placerat justo sed lectus lobortis ut porta nisl porttitor. Vestibulum mi dolor, lacinia molestie gravida at, tempus vitae ligula. Donec eget quam sapien, in viverra eros. Donec pellentesque justo a massa fringilla non vestibulum metus vestibulum. Vestibulum in orci quis felis tempor lacinia. Vivamus ornare ultrices facilisis. Ut hendrerit volutpat vulputate. Morbi condimentum venenatis augue, id porta ipsum vulputate in. Curabitur luctus tempus justo. Vestibulum risus lectus, adipiscing nec condimentum quis, condimentum nec nisl. Aliquam dictum sagittis velit sed iaculis. Morbi tristique augue sit amet nulla pulvinar id facilisis ligula mollis. Nam elit libero, tincidunt ut aliquam at, molestie in quam. Aenean rhoncus vehicula hendrerit.