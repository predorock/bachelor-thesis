\lhead{\emph{Capitolo 4}}
\chapter{Conclusioni}

In questo capitolo si illustreranno quali sono stati i risultati del lavoro svolto e con le opportune valutazioni e commenti. Per confrontare i risultati dello sviluppo multi piattaforma con il modello ibrido, si farà riferimento ai tempi di sviluppo e alla qualità del software prodotto. Inoltre si illustrerà qual'è il trend attuale del mercato delle applicazioni e dei dispositivi mobili, per valutare la scelta delle tecnologie per lo sviluppo. Inoltre si faranno riferimenti a tecnologie web emergenti legate al campo dello sviluppo multi piattaforma.

\section{Considerazioni sullo sviluppo ibrido}
L'esperienza di tirocinio in azienda come Frontend Web Developer, ha condizionato molto la scelta dell'approccio ibrido con tecnologie web. Ho sperimentato personalmente come uno sviluppatore web possa estendere le proprie conoscenze, per sviluppare applicazioni su dispositivi mobili.

\subsection{Quando conviene lo sviluppo ibrido}

-- Fare anche esempi come Whatsapp e Telegram --\\

\subsection{Le tecnologie web, molto di più dei semplici framework}

\section{Risultati sulle tecnologie utilizzate}

-- Vantaggi-- \\
-- Criticità --\\


\section{Ciclo di vita di una app}
-- Statistiche sulle app\\
	-- Stessa App replicata nei store --\\
	-- Durata delle app --\\
-- Aggiornamenti --\\
	-- più piattaforme = più aggiornamenti --\\
-- Mantenimento --\\
	-- Aggiornamento delle versioni --\\
	-- Debug --\\
-- Tutto ovviamente in confronto con lo sviluppo ibrido --

\section{L'aspetto commerciale e le possibili soluzioni}

-- Come può un azienda adottare soluzioni cross platform --\\
-- Azienda\\
	-- Tipi di app sviluppate\\
	-- Approccio ibrido oppure no\\
	-- Scelta in base alla natura delle app

\section{Nuove tecnologie web emergenti}

\subsection{FirefoxOS}

\subsection{WebOS}

\subsection{Node Webkit}