%%%%%%%%%%%%%%%%%%%%%%%%%%%%%%%%%%%%%%%%%
% Masters/Doctoral Thesis 
% LaTeX Template
% Version 1.43 (17/5/14)
%
% This template has been downloaded from:
% http://www.LaTeXTemplates.com
%
% Original authors:
% Steven Gunn 
% http://users.ecs.soton.ac.uk/srg/softwaretools/document/templates/
% and
% Sunil Patel
% http://www.sunilpatel.co.uk/thesis-template/
%
% License:
% CC BY-NC-SA 3.0 (http://creativecommons.org/licenses/by-nc-sa/3.0/)
%
% Note:
% Make sure to edit document variables in the Thesis.cls file
%
%%%%%%%%%%%%%%%%%%%%%%%%%%%%%%%%%%%%%%%%%

%----------------------------------------------------------------------------------------
%	PACKAGES AND OTHER DOCUMENT CONFIGURATIONS
%----------------------------------------------------------------------------------------
\documentclass[11pt, oneside]{Thesis} % The default font size and one-sided printing (no margin offsets)


\usepackage[utf8]{inputenc}
\usepackage[none]{hyphenat}
\usepackage[square, comma, sort&compress]{natbib}
\usepackage{csvsimple}
\usepackage{listings}
\usepackage{color}
\usepackage{hyperref}
\usepackage{wrapfig}
\usepackage{enumitem}
\usepackage{url}

\setlist[description]{style=unboxed,leftmargin=0cm}

\lstnewenvironment{code}[1][]%
  {\noindent\minipage{\linewidth}\medskip   
   \lstset{basicstyle=\ttfamily\footnotesize,frame=single,#1, numbers=left, firstnumber=1,numberfirstline=true}}
  {\endminipage}


\sloppy
% Use the natbib reference package - read up on this to edit the reference style; 
% if you want text (e.g. Smith et al., 2012) for the in-text references (instead of numbers), remove %'numbers' 

\hypersetup{urlcolor=blue, colorlinks=true} % Colors hyperlinks in blue - change to black if annoying
\title{\ttitle} % Defines the thesis title - don't touch this
\begin{document}

\frontmatter % Use roman page numbering style (i, ii, iii, iv...) for the pre-content pages

\setstretch{1.3} % Line spacing of 1.3

% Define the page headers using the FancyHdr package and set up for one-sided printing
\fancyhead{} % Clears all page headers and footers
\rhead{\thepage} % Sets the right side header to show the page number
\lhead{} % Clears the left side page header

\pagestyle{fancy} % Finally, use the "fancy" page style to implement the FancyHdr headers

\newcommand{\HRule}{\rule{\linewidth}{0.5mm}} % New command to make the lines in the title page

% PDF meta-datar
\hypersetup{pdftitle={\ttitle}}
\hypersetup{pdfauthor=\authornames}

%----------------------------------------------------------------------------------------
%	TITLE PAGE
%----------------------------------------------------------------------------------------

\begin{titlepage}
\begin{center}

\begin{figure}
	\begin{center}
		\includegraphics[scale=0.25]{Figures/tesiSCIENZE_TECNOLOGIE.jpg}
	\end{center}
\end{figure}

%\textsc{\LARGE \univname}\\ % University name
%\smallskip
%{\large \facname}\\
\smallskip
\large\textsc{\deptname}\\[1.0cm]

\textsc{\Large Tesi Di Laurea Triennale}\\[0.5cm] % Thesis type

\HRule \\[0.4cm] % Horizontal line
{\huge \bfseries \ttitle}\\[0.4cm] % Thesis title
\HRule \\[1.5cm] % Horizontal line
 
\begin{minipage}{0.4\textwidth}

\begin{flushleft} \large
\emph{Autore:}\\
\authornames             % Author name - remove the \href bracket to remove the link
\end{flushleft}
\end{minipage}
\begin{minipage}{0.4\textwidth}
\begin{flushright} \large
\emph{Relatore:} \\
\supname		 % Supervisor name - remove the \href bracket to remove the link

\emph{Correlatore:} \\
\cosupname		 % Cosupervisor name -
\end{flushright}
\end{minipage}\\[3cm]
 
%\large \textit{A thesis submitted in fulfilment of the requirements\\ for the degree of \degreename}\\[0.3cm] % University requirement text
%\textit{in the}\\[0.4cm]
 \
{\large Anno Accademico 2013/2014}\\[4cm] % Date
%\includegraphics{Logo} % University/department logo - uncomment to place it
 
\vfill
\end{center}

\end{titlepage}
  

\clearpage % Start a new page

%----------------------------------------------------------------------------------------
%	QUOTATION PAGE
%----------------------------------------------------------------------------------------

\pagestyle{empty} % No headers or footers for the following pages

\null\vfill % Add some space to move the quote down the page a bit

\begin{flushleft}

\textit{``L'intelligenza è l'abilità di riuscire ad evitare il lavoro,\\
   portando comunque a compimento i propri compiti.``\\
   Linus Torvalds}
\end{flushleft}

\vfill
\vfill
\vfill

\vfill
\vfill
\vfill
\vfill
\null % Add some space at the bottom to position the quote just right

\clearpage % Start a new page

%----------------------------------------------------------------------------------------
%	ABSTRACT PAGE -- see the design on Thesis.cls
%----------------------------------------------------------------------------------------

\addtotoc{Abstract} % Add the "Abstract" page entry to the Contents

\abstract{\addtocontents{toc}{\vspace{1em}} % Add a gap in the Contents, for aesthetics
I dispositivi mobili, al giorno d’oggi, rivestono un ruolo sempre più importante tanto nelle aziende quando nella nostra vita privata, permettendoci di compiere operazioni e svolgere dei compiti che, fino a qualche anno fa, erano eseguibili solo attraverso un normale PC.
Con il passare del tempo, l’evoluzione tecnologica che ha accompagnato lo sviluppo dei normali PC, ha coinvolto i dispositivi così detti “mobili”. Evoluzione che li ha trasformati da semplici organizer “da tasca” a veri e propri terminali ricchi di funzionalità, discreta potenza di calcolo ma soprattutto di connettività. Quest’ultima caratteristica li ha resi estremamente versatili soprattutto con l'arrivo di applicazioni di tipo aziendale e di produttività personale.

\textit{Dai dati sull’uso dei dif75ferenti device utilizzati per accedere a internet, risulta che il 66,4\% del tempo totale speso online è generato dalla fruizione di internet da mobile e, più in dettaglio, il 	55,7\% del totale dalla fruizione tramite mobile applications.}\\
\cite{stats:audiweb}

Il compito dello sviluppatore, è quello di realizzare applicazioni non solo funzionanti nel senso stretto del termine ma funzionali e usabili su questa tipologia di dispositivi. Inoltre se si vuole far conoscere il prodotto ad un numero sempre più elevato di utenti, ci si deve anche preoccupare che l'applicazione sia disponibile su diverse o tutte le piattaforme, in quanto il mercato delle applicazioni dei dispositivi mobili è abbastanza frammentato.

Esistono diversi modi per sviluppare una applicazione mobile, Nativa, Web o ibrida: ogni approccio ha vantaggi e svantaggi, e la scelta può limitare le opzioni degli strumenti di sviluppo in un secondo momento.

\paragraph*{App native} Sviluppare un'app utilizzando l'interfaccia e il linguaggio di programmazione per un determinato dispositivo e sistema operativo. Ciò può fornire le prestazioni migliori ma richiede una versione differente (costosa) per ogni sistema operativo.

\paragraph*{App Web} Si tratta di sviluppare un sito web che abbia le sembianze di una applicazione vera e propria che possa essere raggiunta tramite un semplice browser web, anche se la frammentazione dell’attuale mercato fa si che i browser comunemente disponibili sui vari dispositivi non supportino in modo uniforme gli attuali standard del web. Con questo approccio inoltre non si ha la possibilità di accedere alle funzionalità del dispositivo come ad esempio fotocamera, contatti, accelerometro, etc \ldots.

\paragraph*{App ibride} Un compromesso tra nativa e Web. Lo sviluppo avviene utilizzando i linguaggi di programmazione Web standard di settore, come HTML5 e JavaScript quindi viene creato un pacchetto di installazione nativa (es. apk file) per la distribuzione tramite app store. Esistono anche altri metodi ma questo è uno di quelli più usati. E' cosi possibile ridurre i costi con il riutilizzo del codice.

L'obiettivo di questo lavoro e quello di trovare un metodo di sviluppo veloce per avere la stessa applicazione disponibile nelle diverse piattaforme. Ovviamente ci sono più percorsi che ci possono portare allo stesso risultato, la scelta della strada da intraprendere dipende da molti fattori che si vedrà essere di vario genere.

Questa tesi si propone l'obiettivo di analizzare le tecnologie sul mercato attuale e proporre un modello di sviluppo rapido di applicazioni multi-piattaforma per dispositivi mobili, attraverso l'utilizzo di tecnologie web, mostrando vantaggi e svantaggi che si possono riscontrare rispetto agli altri metodi di sviluppo.

}

\clearpage % Start a new page
%----------------------------------------------------------------------------------------
%	ACKNOWLEDGEMENTS
%----------------------------------------------------------------------------------------

\setstretch{1.3} % Reset the line-spacing to 1.3 for body text (if it has changed)

\acknowledgements{\addtocontents{toc}{\vspace{1em}} % Add a gap in the Contents, for aesthetics

A mia madre Giusy a mio padre Angelo a mio fratello Dario. Ringrazio la mia famiglia per avermi sostenuto e incoraggiato a seguire le mie passioni nel mio percorso formativo, senza dei quali nulla sarebbe stato possibile.

Ringrazio i compagni universitari più vicini e gli amici del collegio Modena, per l'ottimo tempo trascorso assieme durante gli studi.

Ringrazio la ditta BigThink SRL per l'opportunità di tirocinio data per la mia tesi.

Infine ringrazio pippo, pluto, paperino e topolino per l'aiuto nei test del codice.
}
\clearpage % Start a new page

%----------------------------------------------------------------------------------------
%	LIST OF CONTENTS/FIGURES/TABLES PAGES
%----------------------------------------------------------------------------------------

\pagestyle{fancy} % The page style headers have been "empty" all this time, now use the "fancy" headers as defined before to bring them back

\renewcommand\contentsname{Indice dei Contenuti}
\lhead{\emph{Indice dei Contenuti}} % Set the left side page header to "Contents"
\tableofcontents % Write out the Table of Contents

\renewcommand\listfigurename{Elenco delle Figure}
\lhead{\emph{Elenco delle Figure}} % Set the left side page header to "List of Figures"
\listoffigures % Write out the List of Figures

\renewcommand\listtablename{Elenco delle Tabelle}
\lhead{\emph{Elenco delle Tabelle}} % Set the left side page header to "List of Tables"
\listoftables % Write out the List of Tables

\renewcommand\lstlistlistingname{Elenco degli Esempi}
\lhead{\emph{Elenco degli Esempi}}
\addcontentsline{toc}{chapter}{\lstlistlistingname}
\lstlistoflistings

\renewcommand{\bibname}{Bibliografia}
%----------------------------------------------------------------------------------------
%	ABBREVIATIONS
%----------------------------------------------------------------------------------------

%\clearpage % Start a new page

%\setstretch{1.5} % Set the line spacing to 1.5, this makes the following tables easier to read

%\lhead{\emph{Abbreviations}} % Set the left side page header to "Abbreviations"
%\listofsymbols{ll} % Include a list of Abbreviations (a table of two columns)
%{
%\textbf{ASAP} & \textbf{A}s \textbf{S}oon \textbf{A}s \textbf{P}ossible \\
%\textbf{HTML} & \textbf{H}yper\textbf{T}ext \textbf{M}arkup \textbf{L}anguage \\
%\textbf{CSS} & \textbf{C}ascading \textbf{S}tyle \textbf{S}heet \\
%\textbf{RIA} & \textbf{R}ich \textbf{I}nternet \textbf{A}pplication \\
%\textbf{Acronym} & \textbf{W}hat (it) \textbf{S}tands \textbf{F}or \\
%}

%----------------------------------------------------------------------------------------
%	PHYSICAL CONSTANTS/OTHER DEFINITIONS
%----------------------------------------------------------------------------------------

%\clearpage % Start a new page

%\lhead{\emph{Physical Constants}} % Set the left side page header to "Physical Constants"

%\listofconstants{lrcl} % Include a list of Physical Constants (a four column table)
%{
%Speed of Light & $c$ & $=$ & $2.997\ 924\ 58\times10^{8}\ \mbox{ms}^{-\mbox{s}}$ (exact)\\
% Constant Name & Symbol & = & Constant Value (with units) \\
%}

%----------------------------------------------------------------------------------------
%	SYMBOLS
%----------------------------------------------------------------------------------------

%\clearpage % Start a new page
%\lhead{\emph{Symbols}} % Set the left side page header to "Symbols"

%\listofnomenclature{lll} % Include a list of Symbols (a three column table)
%{
%$a$ & distance & m \\
%$P$ & power & W (Js$^{-1}$) \\
% Symbol & Name & Unit \\

%& & \\ % Gap to separate the Roman symbols from the Greek

%$\omega$ & angular frequency & rads$^{-1}$ \\
% Symbol & Name & Unit \\
%}

%----------------------------------------------------------------------------------------
%	DEDICATION
%----------------------------------------------------------------------------------------

%\setstretch{1.3} % Return the line spacing back to 1.3

%\pagestyle{empty} % Page style needs to be empty for this page

%\dedicatory{For/Dedicated to/To my\ldots} % Dedication text

%\addtocontents{toc}{\vspace{2em}} % Add a gap in the Contents, for aesthetics

%----------------------------------------------------------------------------------------
%	THESIS CONTENT - CHAPTERS
%----------------------------------------------------------------------------------------

\mainmatter % Begin numeric (1,2,3...) page numbering

\pagestyle{fancy} % Return the page headers back to the "fancy" style

% Include the chapters of the thesis as separate files from the Chapters folder
% Uncomment the lines as you write the chapters

% Chapter Template

\chapter{HTML, CSS, Javascript - L'evoluzione dei linguaggi web} % Main chapter title

\label{ChapterX} % Change X to a consecutive number; for referencing this chapter elsewhere, use \ref{ChapterX}

\lhead{Capitolo 2. \emph{HTML, CSS, Javascript - L'evoluzione dei linguaggi web}} % Change X to a consecutive number; this is for the header on each page - perhaps a shortened title

Linguaggi come l'HTML(Hypertext Markup Language), CSS(Cascading Style Sheet) e Javascript sono i principali strumenti con quali costruiamo pagine web, ebbene la continua evoluzione di questi linguaggi ha portato il classico sito web, a somigliare molto ad una applicazione vera e propria. 
Questo però non sarebbe stato possible se  questi tre linguaggi on si fosssero evoluti parallelamente.
\section{L'HTML - Da descrittore di documenti a descrittore di interfacce}
Nato in origine per la descrizione dei primi documenti reperibili sul web, lo standard HTML è diventato ben presto HTML5, con la possibilità di offrire all'utente un' interfaccia dinamica con cui interagire.
Guardando ai giorni nostri le caratteristiche più significative che vogliamo far notare sono sicuramente il supporto ai contenuti multimediali e la possibilità di creare presentazioni grafiche molto accativanti
arricchendo il codice con i CSS.

\section{CSS - Dallo stile dei documenti alle trasformazioni vettoriali}
Il linguaggio CSS è servito fin dalla sua origine alla descrizione dello stile di un documento HTML per separarlo dal suo contenuto in modo che il codice risultasse più leggibile e riutilizzabile.
L’introduzione del CSS si è resa necessaria per separare i contenuti dalla formattazione e permettere una programmazione più chiara e facile da utilizzare, sia per gli autori delle pagine HTML che per gli utenti, garantendo contemporaneamente anche il riuso di codice ed una sua più facile manutenibilità.
L'ultimo standard CSS3 ha introdotto inoltre nuve librerie grafiche che si appoggiano sul motore di rendering del browser(Blink, Trident, Presto, Gecko) che permettono animazioni e modifiche molto avanzate dei componenti web.
L'unica nota negativa è che ogni motore di rendering utilizza prefissi diversi nelle varie funzioni (-moz ,-webkit). Quindi se si vuole la stessa elaborazione su browser differenti bisogna scrivere la stessa operazione con prefissi differenti.

\section{Javascript - Dalle animazioni ai potenti Framework}
\subsection{Descrizione}
Javascript è un linguaggio di scripting lato client che viene interpetato dal browser. Permette di aggiungere delle animazioni alle nostre pagine web ma negli ultimi anni molte espansioni gli hanno permesso di estendersi in moltissimi altri usi.
\subsection{Storia}
Nel 1995 Netscape decise di dotare il proprio browser di un linguaggio di scripting che permettesse ai web designer di interagire con i diversi oggetti della pagina. Brendan Eich(ora in Mozilla) venne incaricato del progetto e creò la prima versione chiamata Mocha. Il fondatore di Netscape Marc Andreessen cambiò il nome nel 1995 in Live script, data appunto la dinamicità ottenuta nelle pagine web tramite il linguaggio. Alla fine dell'anno venne presentato il linguaggio con il nome di Javascript in quanto Netscape lo aveva creato principalmente per l'interazione con le applet Java.
Successivamente data la risposta di Microsoft con un linguaggio praticamente uguale ma chiamato JScript la le aziende SUN, Netscape e Microsoft decisero di chiedere a ECMA la standardizzare del linguaggio Javascript che venne ri-batezzato con il nome di ECMAScript.
\subsection{I Cambiamenti}
\subsection{I Framework attuali}

%----------------------------------------------------------------------------------------
%	SECTION 2
%----------------------------------------------------------------------------------------

\section{Main Section 2}

Sed ullamcorper quam eu nisl interdum at interdum enim egestas. Aliquam placerat justo sed lectus lobortis ut porta nisl porttitor. Vestibulum mi dolor, lacinia molestie gravida at, tempus vitae ligula. Donec eget quam sapien, in viverra eros. Donec pellentesque justo a massa fringilla non vestibulum metus vestibulum. Vestibulum in orci quis felis tempor lacinia. Vivamus ornare ultrices facilisis. Ut hendrerit volutpat vulputate. Morbi condimentum venenatis augue, id porta ipsum vulputate in. Curabitur luctus tempus justo. Vestibulum risus lectus, adipiscing nec condimentum quis, condimentum nec nisl. Aliquam dictum sagittis velit sed iaculis. Morbi tristique augue sit amet nulla pulvinar id facilisis ligula mollis. Nam elit libero, tincidunt ut aliquam at, molestie in quam. Aenean rhoncus vehicula hendrerit.
\chapter{Metodi e Modelli per lo sviluppo di applicazioni Mobile}
La scelta di seguire un determinato schema progettuale di una applicazione è stata fatta in base alle conoscenze apprese durante la mia esperienza di tirocinio in azienda, durante la quale non mi sono occupato dell'intera progettazione dell'applicazione, mi occupavo solo dello sviluppo \emph{frontend} di questa, ma per uno sviluppatore è necessario conoscere l'intera dinamica che si svolge all'interno di essa anche se non andrà a intervenire su determinate parti.\\

La politica che è stata intrapresa in azienda è stata quella della \emph{Separation of Concernes}(SoC), che in italiano si traduce in \emph{Separazione dei Compiti}. Si tratta di un principio di design dell'applicazione per dividere l'applicazione in sezioni distinte, e ad ogni sezione assegnare un particolare compito o risoluzione di un problema. Questa scelta progettuale introduce il concetto di \emph{modulo} di una applicazione, ovvero l'applicazione poi dipenderà dai moduli che andremo a creare e ad unire tra di loro. I moduli ragionevolmente saranno indipendenti uno dall'altro in modo tale da garantirne l'integrità e il loro re-utilizzo.\cite{wiki:soc}\\
Il concetto appena visto astrae molto da una applicazione vera e propria, sta a chi la progetta decidere con che granularità  e se è veramente necessario applicarlo. Dato che parliamo di tecnologie web, un chiaro esempio si separazione dei compiti è quello della struttura di una pagina web, divisa in HTML per la struttura, CSS per lo stile, e Javascript per la logica e comportamento.\\

In questo capitolo andremo a vedere in che modo i compiti si possono separare all'interno di una applicazione, e quali strutture e/o design pattern possono risultare utili per lo sviluppo.
 
\section{Design Patterns}

Quando si progetta una applicazione risulta molto utile utilizzare schemi architetturali e metodologici che rendono da un lato l'applicazione efficiente dall'altro una scrittura del codice molto chiara e modulare facile da correggere nel caso di eventuali errori. I design pattern vengono in aiuto a questa esigenza del programmatore.

\emph{In informatica, nell'ambito dell'ingegneria del software, un design pattern (traducibile in lingua italiana come schema progettuale, schema di progettazione, schema architetturale), è un concetto che può essere definito "una soluzione progettuale generale ad un problema ricorrente". Si tratta di una descrizione o modello logico da applicare per la risoluzione di un problema che può presentarsi in diverse situazioni durante le fasi di progettazione e sviluppo del software, ancor prima della definizione dell'algoritmo risolutivo della parte computazionale.}
\hspace*{\fill}\cite{wiki:design_pattern} 

Il problema ricorrente che si ha nello sviluppo di applicazioni e quello della decisione di dove debbano stare determinati compiti / operazioni / sezioni all'interno dell'applicazione. Ad esempio la separazione della parte dell'interfaccia da quella dedicata alla manipolazione dei dati da quella dedicata alla memorizzazione. Dividere i vari compiti di una applicazione può risultare molto produttivo, in quanto si possono sviluppare in parallelo diverse parti dell'applicazione senza influire sulle altre e volendo con la possibilità di utilizzare tecnologie differenti.

Durante il periodo di tirocinio presso l'azienda \emph{BigThink SRL} per la creazione di Web Applications sono stati utilizzati pattern come : Client-Server, SoC, Frontend e Backend gli altri design pattern che verranno introdotti sono necessari per comprendere meglio alcune tecnologie che verranno utilizzate.
 
\subsection{Frontend e Backend}
Queste due parole sono spesso usate in informatica in molti ambiti, nel contesto specifico dell'applicazione \texttt{frontend}(in italiano parte davanti) denota quella parte dell'applicazione responsabile di gestire l'interfaccia utente e i dati provenienti da essa, mentre \texttt{backend}(in italiano parte dietro) indica la sezione dell'applicazione dedita alla gestione dei dati provenienti dalla parte frontend. L'interazione che hanno le due parti è un chiaro esempio di interfaccia.\\
\textbf{Frontend:} questa è la parte caratteristica dell'applicazione, in quanto ne definisce il comportamento e l'aspetto, determinando la logica con cui si evolverà al rapporto con l'utente. A differenza della parte \emph{backend} questa non definisce nessuna manipolazione dei dati ma solo la loro rappresentazione(vista) determinando transizioni e/o animazioni.
Nella parte \emph{frontend} è inclusa anche la fase di definizione estetica dell'interfaccia, ma spesso questa spetta a una figura professionale distinta atta "vestire" l'applicazione.
\textbf{Backend:} questa parte è completamente diversa dalla prima, in quanto definisce la manipolazione dei dati all'interno dell'applicazione ma non da nessuna informazione caratteristica di essa. In particolare fornisce dei servizi ai quali la parte \emph{frontend} può accedere e recuperare/fornire dei dati, come ad esempio l'autenticazione di un utente a un servizio. Tutta la gestione dei dati che viene fatta da questa parte, viene oscurata alla parte \emph{frontend} per garantire un servizio di sicurezza molto elementare, in modo tale che se l'utente inserisce dei dati errati che vengono passati dalla parte \emph{frontend} a quella \emph{backend}, in modo errato, nessuna operazione verrà eseguita e l'integrità dei dati preservata.\\

A livello professionale molti sviluppatori si identificano appunto come \emph{frontend} e/o \emph{backend} developer per appunto identificarsi specializzati nello sviluppo di una parte specifica dell'applicazione.
Il vantaggio di usare un approccio di questo tipo sta nel fatto che la parte \emph{frontend} è l'unica specifica per una data applicazione. Se la parte \emph{backend} viene progettata bene, e possibile utilizzare i servizi che fornisce anche in futuro da altre applicazioni, pur seguendo il protocollo che richiede. Inoltre a livello professionale si può procedere parallelamente nello sviluppo delle parti in modo tale da ottimizzare i tempi, pur seguendo uno schema stabilito a priori.\\

\subsection{Pattern Client-Server}
\textbf{Definizione:} Il modello client-server è un modo per strutturare applicazioni distribuite che distingue due parti di un processo di comunicazione, la prima che fornisce una risorsa e/o un servizio chiamata server, la seconda che analogamente li può richiedere, chiamata client. La comunicazione in generale avviene attraverso la rete, ed è il client a iniziarla. Il compito del server è quello di predisporre le risorse che ha ai vari client che li chiedono, rimanendo appunto "in ascolto", il client invece non condivide le risorse con altri, può solo interagire con il server\cite{wiki:cliserv}.\\

Come si può già intuire le parti che verranno prese in questo modello saranno rispettivamente quella del \emph{frontend} e quella del \emph{backend}. Specificatamente nell'ambito delle applicazioni i client saranno le varie istanze della parte \emph{frontend} dell'applicazione sui vari dispositivi, mentre il server, fornitore di servizi, conterrà la parte di \emph{backend}.

\subsection{Mustache}

In ambito web può essere utile un sistema di template che aiuti a separare la logica dalla presentazione dei dati(metodologia molto in voga tra le tecnologie web).

\emph{Mustache è un web template system con una implementazione disponibile in molti linguaggi(sarà utili quella in Javascript). Mustache è descritto come un sistema senza logica, in quanto manca qualsiasi dichiarazione di controllo del flusso esplicita, ovvero istruzioni condizionali e di iterazione. Il tutto può essere implementato tramite un sistema di tag, liste procedurali e lambda}
\hspace*{\fill}\cite{wiki:mustache}

\subsection{MVC Pattern}
\label{sec:MVC}
\emph{Il Model-View-Controller pattern (MVC) in informatica, è un pattern architetturale molto diffuso nello sviluppo di sistemi software, in particolare nell'ambito della programmazione orientata agli oggetti, in grado di separare la logica di presentazione dei dati dalla logica di business.}

Il pattern è basato sulla separazione dei compiti fra i componenti software che interpretano tre ruoli principali:
\begin{description}
\item[model] fornisce i metodi per accedere ai dati utili all'applicazione;
\item[view] visualizza i dati contenuti nel model e si occupa dell'interazione con utenti e agenti;
\item[controller] riceve i comandi dell'utente (in genere attraverso il view) e li attua modificando lo stato degli altri due componenti.
\end{description}
Questo schema, fra l'altro, implica anche la tradizionale separazione fra la logica applicativa (in questo contesto spesso chiamata "logica di business"), a carico del controller e del model, e l'interfaccia utente a carico del view.
\hspace*{\fill}\cite{wiki:mvc}

\subsection{MVVC Pattern}
\emph{Il Model-View-ViewModel è un pattern architetturale basato sul pattern MVC e MVP che tenta di separare più chiaramente lo sviluppo di interfacce utente (UI) da quello della logica di business e il comportamento in un'applicazione . A tal fine, molte implementazioni di questo modello fanno uso di associazioni dati dichiarativa(data bindings) per consentire una separazione di lavoro su Vista da altri livelli.}

Questo facilita l'interfaccia utente e lo sviluppo che si verificano quasi contemporaneamente all'interno dello stesso codice. Gli sviluppatori dell'interfaccia utente scrivono associazioni verso il ViewModel nel documento di markup (HTML) mentre il Model e il ViewModel sono mantenuti da altri sviluppatori che lavorano sulla logica dell'applicazione(business logic).

\hspace*{\fill}\cite{book:mvvm}
\section{API}

\emph{In informatica le API(Application Programming Interface) sono un insieme di operazioni, protocolli, strumenti per lo sviluppo del software. Una API rappresenta una specifica componente del software in termini di operazione, input, output e tipi soggiacenti. Una API definisce funzionalità totalmente indipendenti dalla loro rispettiva implementazione, il che consente di poter variare rispettivamente l'implementazione e la definizione senza che una influisca sull'altra. Una buona API rende più facile lo sviluppo del software fornendo vari "mattoni" con cui poter sviluppare il software, il ruolo del programmatore è quello di unire i vari blocchi richiesti.}
\hspace*{\fill}\cite{wiki:api}

Un primo esempio di API è già stato menzionato quando si è parlato dei framework per lo sviluppo ibrido. In quel caso le API erano fornite tramite Javascript in modo tale che potessero essere interpretate all'interno di un contenitore nativo che potesse interpretare linguaggi web(come un browser). In questo caso i blocchi di cui si parla sono le rispettive API che consentono di comunicare con le funzionalità del dispositivo come ad esempio la fotocamera, il gps, la vibrazione, l'accelerometro ecc\ldots.

L'utilizzo di API è stato fatto durante il mio tirocinio aziendale per lo scambio di dati tra la parte Frontend e Backend di Web Application, in particolare sono state utilizzate delle API REST.

\subsection{API Rest}

\textbf{Re}presentational \textbf{S}tate \textbf{T}ransfer (REST) è uno stile di architettura software per sistemi distribuiti. Un sistema conforme ai principi di design REST è detto RESTful. Il Web è l'esempio più importante e più famoso di sistema distribuito che realizza i principi REST. D'altra parte non è un caso. Infatti Roy Fielding, il padre di REST, introdusse per la prima volta l'espressione Representational State Transfer nella sua tesi di dottorato per indicare le caratteristiche delle parti meglio progettate del Web.
Il termine REST è diventato più comune negli ultimi anni con l'esplosione del Web 2.0 e il proliferare di servizi web.
In poche parole un'architettura RESTful è di tipo client / server. I client fanno richieste ai server che a loro volta processano le richieste e restituiscono una risposta. Richieste e risposte contengono delle rappresentazioni di risorse. Una \textbf{risorsa} è un concetto significativo per il dominio di riferimento (ad esempio l'articolo di un blog o il profilo utente di un social network) che può essere identificato attraverso un indirizzo. Una \textbf{rappresentazione} è una descrizione dello stato corrente o voluto di una risorsa.

\begin{figure}[htbp]
  \centering
    \includegraphics[scale=0.75]{Figures/tab-rest.jpg} 
    \rule{35em}{0.5pt}
  \caption[Vincoli REST]{Vincoli che una applicazione deve rispettare per essere considerata RESTful}
  \label{fig:REST Rules}
\end{figure}

Lo stile architetturale REST descrive 6 vincoli (cinque più uno facoltativo) che devono essere rispettati da un sistema affinché possa essere definito RESTful (tabella \ref{fig:REST Rules}). Poiche’ REST è uno stile e non una tecnologia e nemmeno una lista di raccomandazioni tecniche, i dettagli su come questi vincoli debbano essere implementati sono lasciati ai singoli sviluppatori. Ad esempio, l'Application Layer sul quale costruire il sistema non deve essere per forza HTTP, anche se HTTP è quasi sempre la scelta più naturale. Il formato delle rappresentazioni delle risorse spesso è XML o JSON, ma potrebbe essere anche PDF. Infine REST non è uno stile architetturale solo per le web application, ma è molto più generale. Se i vincoli REST sono rispettati, l’architettura software di un robot (server) controllato da uno o più computer (client) dove le risorse sono l’hardware del robot (sensori e motori) può essere a tutti gli effetti considerata RESTful.
L'esperienza ha mostrato che sistemi RESTful hanno delle desiderabili proprietà emergenti, come ad esempio la scalabilità e la modificabilità.
Un esempio di organizzazione delle risorse nel protocollo REST:

\begin{lstlisting}[language=php, caption = {Esempio di API REST generica}, label = {lst:genericRESTAPI}]
	/{tipo di risorsa}/{id della risorsa}/{azione}
\end{lstlisting}

applicando l'esempio nel contesto dei blog:

\begin{lstlisting}[language=php, caption = {Esempio di API REST nel caso di un ipotetico blog} , 
				   lablel = {lst:blogRESTAPI}]
	/posts/1/edit oppure /posts/1/view
\end{lstlisting}

\subsection{Chiamate XHR}

 
\lhead{\emph{Capitolo 3}}
\chapter{La soluzione proposta}
In questo capitolo discuteremo la soluzione intrapresa per lo sviluppo veloce di applicazioni mobile multi-piattaforma con le tecnologie scelte e le scelte progettuali intraprese. 
\section{Le Tecnologie Web}
Le tecnologie web sono sempre più utilizzate per lo sviluppo di applicazioni mobile multi-piattaforma, molti framework come \texttt{Ionic}, \texttt{Foundation}, \texttt{Lungo} utilizzano già HTML5, CSS, Javascript per le loro librerie, ottimizzate per dispositivi mobili.\\
In particolare questi framewrok mettono a disposizione una serie di componenti già pronti come ad esempio bottoni, menù laterali, form, liste, tabelle, icone, etc\dots e consentono di abbinarli tra di loro come meglio si ritene, con la possibilità di personalizzarli ulteriormente.\\
\subsection{Ionic}

Ionic offre componenti e librerie pensati per uno sviluppo ibrido dell'applicazione, inoltre è stato sviluppato riducendo al minimo la manipolazione del DOM\footnote{Il Document Object Model (spesso abbreviato come DOM), letteralmente modello a oggetti del documento, è una forma di rappresentazione dei documenti strutturati come modello orientato agli oggetti.\cite{wiki:dom} gli oggetti delle pagine web, ad esempio con delle animazioni, introduciamo della computazione aggiuntiva al nostro browser che può rallentarne le prestazioni} garantendo performance molto competitive.\\
I componenti di Ionic vengono ovviamente strutturati tramite HTML5, aggiungendo classi css specifiche del framework. Inoltre 
Ionic fornisce uno strumento molto interessante da linea di comando che permette di scaricare diversi modelli di applicazione già pronti in modo da non dover tutte le volte configurare da capo la nostra applicazione con il framework. Inoltre ci predispone i file in una maniera logica ben precisa cosicché se volessimo in un futuro aggiungere codice e/o altre librerie possiamo farlo con molta facilità.
I componenti non sono altro che elementi del linguaggio HTML, vengono forniti con una serie di classi CSS caratteristiche del framework, pensate in modo da essere componibili tra di loro. Ionic fornisce inoltre dei tag propri del framework che possono includere elementi più complessi come ad esempio menù laterali.

Il cuore di Ionic e tutte le sue funzionlità sono state sviluppate in AngularJS (framewrok che vedremo nel capitolo successivo) il che rende questo framework ancora più versatile e potente.
\subsection{AngularJS}
AngularJS è un framework Javascript ideato da \emph{Google} per rendere dinamiche le pagine web. E' stato pensato con un modello di sviluppo \emph{Model View Controlled} quindi l'applicazione che si andrà a sviluppare avrà un serie di \emph{view} dinamiche, a loro volta gestite da entità chiamate \emph{controller}.\\
Una delle caratteristiche che lo rende un framework molto potente da usare è sicuramente il \texttt{Data Bindign}. Il Data Binding è un modo per aggiornare dati in una vista ogni volta che questo cambia senza bisogno di aggiornare la pagina o di modificare il DOM. La cosa è reciproca, se i dati di modificano nella \emph{view} (come potrebbe accadere nel caso di un form) questi sono automaticamente aggiornati dinamicamente all'interno della nostra applicazione.\\
Abbiamo parlato prima di \emph{controller}, sostanzialmente sono delle entità che controllano il comportamento degli elementi della vista. All'interno dei \emph{controller} possiamo definirci tutti i metodi e le funzioni che desideriamo inserire all'interno della vista.\\
AngularJS consente inoltre di creare i propri tag HTML personalizzati, con la possibilità di aggiungere nuovi attributi opzionali e non. Il punto di forza delle direttive e la loro riusabilità, in quanto , se scritte bene, diventano tag e/o attributi riutilizzabili come del comune codice HTML.\\
Un altro punto di forza è sicuramente la \texttt{Dependency Injection} (verrà spiegata in dettaglio nei capitoli successivi). Questa caratteristica in AngularJS consiste nel descrivere come è connessa la mia applicazione, non abbiamo bisogno di un metodo \emph{main()}, come ad esempio si usa in \emph{Java}, ma possiamo decidere a priori di quali moduli è composta la nostra applicazione.
\begin{center}
\emph{This means that any component which does not fit your needs can easily be replaced.} 
\end{center}
Come abbiamo appena detto, una applicazione in AngularJS può essere facilmente divisa in moduli. Questa caratteristica garantisce una riduzione della complessità del codice che si andrebbe a scrivere, inoltre come il suo inventore Minsko Hevery ripete nella maggior parte delle sue conferenze, AngularJS è stato pensato per essere facilmente testabile.

\section{Cordova}
\subsection{ngCordova}
\section{API Rest}

\lhead{\emph{Capitolo 4}}
\chapter{Conclusioni}

\section{Considerazioni sullo sviluppo ibrido}

-- Fare anche esempi come Whatsapp e Telegram --\\

\section{Ciclo di vita di una app}
-- Statistiche sulle app\\
	-- Stessa App replicata nei store --\\
	-- Durata delle app --\\
-- Aggiornamenti --\\
	-- più piattaforme = più aggiornamenti --\\
-- Mantenimento --\\
	-- Aggiornamento delle versioni --\\
	-- Debug --\\
-- Tutto ovviamente in confronto con lo sviluppo ibrido --

\section{Verso il futuro} 
%\input{Chapters/Chapter5} 
%\input{Chapters/Chapter6} 
%\input{Chapters/Chapter7} 

%----------------------------------------------------------------------------------------
%	THESIS CONTENT - APPENDICES
%----------------------------------------------------------------------------------------

\addtocontents{toc}{\vspace{2em}} % Add a gap in the Contents, for aesthetics

\appendix % Cue to tell LaTeX that the following 'chapters' are Appendices

% Include the appendices of the thesis as separate files from the Appendices folder
% Uncomment the lines as you write the Appendices

% Appendix A
\lhead{\emph{Appendice}}
\chapter{Appendice Argomenti} % Main appendix title

\label{Appendice} % For referencing this appendix elsewhere, use \ref{AppendixA}
% This is for the header on each page - perhaps a shortened title
\section{HTML5}
\label{app:html5}
\section{CSS}
\label{app:css}
\section{Javascript}
\label{app:js}
\section{Dependency Injection}
\label{app:DepInj}
\section{Inversion Of Control}
\label{app:ioc}
\section{Future / Promises}
\label{app:future}
%\input{Appendices/AppendixB}
%\input{Appendices/AppendixC}

\addtocontents{toc}{\vspace{2em}} % Add a gap in the Contents, for aesthetics

\backmatter

%----------------------------------------------------------------------------------------
%	BIBLIOGRAPHY
%----------------------------------------------------------------------------------------
\label{Bibliografia}

\lhead{\emph{Bibliografia}} % Change the page header to say "Bibliography"
\bibliographystyle{unsrt} % Use the "unsrtnat" BibTeX style for formatting the Bibliography

\bibliography{Bibliography} % The references (bibliography) information are stored in the file named "Bibliography.bib"


\end{document}