\chapter{Stato dell'arte e obiettivi della tesi}
In questo capitolo si va a presentare il lavoro di ricerca sulle tecnologie web per lo sviluppo di applicazioni cross-platform svoltosi presso la ditta BigThink SRL. La fine del capitolo descriverà gli obiettivi di questa ricerca.
\section{Lo stato dell'arte}
I dispositivi mobili, al giorno d’oggi, rivestono un ruolo sempre più importante tanto nelle aziende quando nella nostra vita privata, permettendoci di compiere operazioni e svolgere dei compiti che, fino a qualche anno fa, erano eseguibili solo attraverso un normale PC.
Non è raro oggi incontrare, in un comune bar, un rappresentante che acquisisce gli ordini direttamente su un Pocket PC e li invia direttamente in azienda sfruttando una connessione GPRS o UMTS, riducendo i tempi di lavoro e, di conseguenza, aumentando potenzialmente il volume d’affari.
Il nostro compito, in quanto sviluppatori, è quello di realizzare applicazioni non solo funzionanti nel senso stretto del termine ma funzionali e usabili su questa tipologia di dispositivi.
Con il passare del tempo, l’evoluzione tecnologica che ha accompagnato lo sviluppo dei normali PC, ha coinvolto i dispositivi così detti “mobili”. Evoluzione che li ha trasformati da semplici organizer “da tasca” a veri e propri terminali ricchi di funzionalità, discreta potenza di calcolo ma soprattutto di connettività. Quest’ultima caratteristica li ha resi estremamente versatili soprattutto per applicazione di tipo aziendale e di produttività personale.

\section{Big Think SRL}
L'azienda Big Think SRL di Daniele Ghidoli è una startup che si occupa della creazione di applicazioni web. All'inizio della mia esperienza ho dovuto imparare come usare gli strumenti che facevano parte dell'ambiente di sviluppo interno. Una volta apprese le conoscenze necessarie per svolgere il lavoro richiesto ho dedicato parte del tempo nella ricerca della mia tesi. La prima fase di apprendimento / addestramento mi è servita per avere buone basi dalle quali partire per lo sviluppo della mia tesi.
\section{Obiettivi del lavoro}
Ho voluto pormi gli obiettivi di questa ricerca dando uno sguardo alla realtà del mondo del lavoro che si sta rapidamente evolvendo in questa fetta del mercato delle applicazioni web.
\subsection{La richiesta del mercato}
\subsubsection{Ciclo di vita di una app}
Ci siamo posti la domanda di quanto debba durare la vita di una applicazione in generale. Sicuramente se si vuole mantenere la propria applicazione all'interno dello store di riferimento bisogna tenere conto dell'evolversi delle tecnologie e quindi preoccuparsi degli aggiornamenti alle versioni dei linguaggi funzionanti. 
\subsubsection{Più mercati su cui competere}
Inoltre dobbiamo tenere conto che ci sono diversi mercati di applicazioni su cui competere, i quali utilizzano linguaggi differenti che a sua volta devono essere mantenuti e aggiornati.
Un altro problema legato alla concorrenza su mercati differenti, è il costo di produzione delle diverse applicazioni in linguaggi diversi; implica avere a disposizione più sviluppatori con diverse competenze e quindi ad aumentare le spese di produzione.
\subsection{Modelli di sviluppo}
Come abbiamo visto dai dati precedenti il traffico internet proveniente dai dispositivi mobili e nettamente in crescita. Lo scopo di questa tesi si occuperà di ricercare soluzioni per lo sviluppo di applicazioni multi-piattaforma per dispositivi mobili.

Per fare ciò abbiamo bisogno di vedere quali modelli di sviluppo si adattano alle nostre esigenze e competenze.

\subsubsection{Sviluppo Nativo}
Nello sviluppo nativo di una applicazione dobbiamo tenere conto come detto in precedenza che esistono più mercati(Store di applicazioni) dove una eventuale azienda potrebbe decidere di concentrare la sua applicazione. Ecco le principali piattaforme di sviluppo al momento:

\begin{itemize}
\item Android, Java
\item iOS, Objective C / Swift
\item Windows Phone, .NET
\item Blackberry, C++/Qt
\item FirefoxOS, HTML5, CSS, Javascript
\item Tizen, HTML5, CSS, Javascript
\end{itemize}

\subsubsection{Le web app}
Una realtà che si sta via via prendendo sempre più piede nello sviluppo delle applicazioni e quella delle Web App. Si possono definire Rich Internet Application (RIA) ed hanno tutte le caratteristiche di un programma che normalmente è in esecuzione sul nostro desktop con la differenza che anziché essere installate nel sistema operativo vengono eseguite nel browser web. 

I vantaggi di questo approccio sono: abbiamo la nostra Applicazione in esecuzione in una Sandbox separata dal resto del sistema operativo ovvero il browser web(che può essere generico, non ne serve uno specifico), non dobbiamo preoccuparci degli aggiornamenti (basta aggiornare la pagina nel browser), non dobbiamo imparare a sviluppare in linguaggi e paradigmi diversi, ci bastano HTML5, CSS e Javascript.

Dall'altro lato della medaglia siamo limitati sotto certi aspetti: l'accesso alle funzionalità del sistema operativo è molto limitato, inoltre la memorizzazione di dati risulta molto difficile da implementare(se cancelliamo la cache del browser perdiamo tutti i dati della nostra applicazione). Ma il più grande deficit è che la Web App per funzionare deve essere sempre connessa ad Internet. 
\subsection{La ricerca della soluzione}  
Abbiamo visto alcuni dei vantaggi e degli svantaggi dei vari tipi di approccio allo sviluppo, è abbiamo questa situazione:
\begin{center}
	\begin{tabular}{| l || c | c | }
		\hline
							&	Native App	&	Web App       \\
		\hline
		\hline
			Cross Platform?	&	No			&	Si            \\
		\hline
			Accesso alle api 
			del sistema 
			operativo?		&	Si			&	Più no che si \\
		\hline
			Store di dati	&	Si			&	Più no che si \\
		\hline
			Connessione 
			obbligatoria ?	&	No			&	Si 			  \\
		\hline
			Velocità di 
			esecuzione	    &	Ottima	 	& 	Buona \\
		\hline
			Tecnologia di 
			sviluppo		&	Objective C, 
			  					Java, C++,..	&	HTML, CSS, Javascript \\
		\hline
			Update 
			installabili?	&	Si			&	Non servono!! \\	
		\hline		
	\end{tabular}
\end{center}
Continuando le ricerche ho scoperto che la scelta migliore non è nei due approcci, ma sta nel mezzo.
Le \emph{Native Web Application} sono la soluzione discussa in questa tesi, e combinano i fattori positivi dei 2 approcci appena visti, con qualche cambiamento.
 